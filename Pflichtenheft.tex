%%%%%%% EINLEITUNG %%%%%%%
% Das Lastenheft beschreibt das Grobkonzept eines Projektes aus sicht des Auftraggebers.
% Es beschreibt lediglich die Basisanforderungen und geht nicht tief ins Detail.
%
% In dem Pflichtenheft wird daher von dem Projektbearbeiter dargelegt, 
% wie er die Aufgabenstellung interpretiert und in welchen Schritten er im Detail
% die Aufgabe abarbeiten m�chte.
% Das Pflichtenheft muss mit dem Auftraggeber (Betreuer) abgestimmt werden und
% gilt danach als verbindliche Grundlage f�r die Bewertung der in dem Projekt erbrachten Leistungen.
%
%%%%%%% BENUTZUNG DIESER VORLAGE %%%%%%%
%
% In dieser Vorlage gibt es zu jedem Abschnitt einen Kommentar mit einer kurzen Erkl�rung.
% Darunter ist stehts ein konkretes Beispiel f�r ein fiktives Projekt abgegeben.
%
% Bei der Erstellung des eignen Pflichtenheftes sollten die eingetragenen Beispiele
% vollst�ndig durch entsprechende Eintragungen f�r das eigenen Projekt ersetzt werden.
%


%%%%%%% PR�AMBEL %%%%%%%
% Dokumentenklasse
\documentclass[ngerman,openany]{scrartcl}
\usepackage[german]{babel}
\usepackage[latin1]{inputenc}
\usepackage[T1]{fontenc}
 
% Pakete
\usepackage{ae}
\usepackage{tabularx}

% Titel
\title{Pflichtenheft \\ Minecraft-Projekt}
\subtitle{Otto-von-Guericke Universit�t Magdeburg, Wintersemester 2016/17}
\author{Erik Koltermann, Angelika Ophagen \\ Betreuer: Alexander Dockhorn
}

\date{26.12.2016}

%%%%%%% DOKUMENT %%%%%%%
\begin{document}
\maketitle

\tableofcontents

\section{Projektziel}
% Hier soll das Ziel des Projektes in eigenen Worten beschrieben werden.
% Dabei sollte auf die Bedeutung und den Nutzen des Projektergebnisses eingegangen werden.
Zielstellung des Softwareprojekts ist es wichtige Bestandteile autonomer Softwareagenten f�r die Spielumgebung Minecraft zu entwickeln. Hierbei wird auf Microsofts Malmo Plattform zur�ckgegriffen, welche als Schnittstellte f�r frei programmierbare Bots mit der Spielumgebung dient.
\\
Fokus des Projektes wird sein, dass ein ''Steve''-Bot in Minecraft selbst�ndig durch Labyrinthe zu einem Ziel man�vrieren. Er soll einen m�glichst kurzen Weg und eine m�glichst schnelle Zeit bis zur Zielfindung ben�tigen. Dabei soll er die Umgebung erkennen und ermitteln, welche von Interaktionsm�glichkeiten ihm geboten sind.
Zur Erstellung einer Welt mit Labyrinth wird zus�tzlich ein Generator programmiert, der dem Bot eine Umgebung schafft.

\section{Anforderungen}
% Hier sollen die Anforderungen an das Projektergebnis beschrieben werden. 
% Es soll zwischen unbedingt erforderlichen Muss-Anforderungen und optionalen Kann-Anforderungen unterschieden werden.

\begin{enumerate}
\item Labyrinthgenerator
\item Programm f�r den Bot
\item Programm einer Versuchsanordnung zur G�tepr�fung des Bot
\item Programmteile zur Versuchsauswertung der G�tepr�fung
\end{enumerate}

\subsection{Muss-Anforderungen}
% Hier soll eine vollst�ndige und nachpr�fbare Beschreibung einer Anforderung an das Projektergebnis beschrieben werden.
\subsubsection{Labyrinthgenerator}
Die zu bezwingenden Labyrinthe sollen 33 x 33 gro� sein und Wege von einem Minecraft-Block Breite aufweisen. Die W�nde sollen zwei Blocks hoch sein und dem Bot das �berspringen unm�glich machen. Es soll im Labyrinth einen k�rzesten Weg zum Ziel geben, aber auch Umwege und Sackgassen.\\
Die Koordinaten des Labyrinthes werden 0-basiert angegeben, der Ursprung ist links unten. Die Koordinaten mit den Koordinaten 0 oder 32 in x oder y bleiben frei. Die Koordinaten mit 1 oder 31 in x oder y sind die Au�enmauern. Es bleibt in der Au�enmauer eine L�cke bei (x,y) = (2, 1). Der Startplatz des Bot ist stets (2, 0) und seine Ausrichtung ist (0,1), wobei dies dem Blick Richtung Westen in Minecraft entspricht.
\\
Das Ziel soll in der Mitte sein und es soll zum n�chsten Pfad so viel Abstand sein, dass nur beim Finden des Sollweges des Ziel als gefunden gilt. Bei einem Detektionsradius von � Minecraft-Block bedeutet dies, dass alle anderen Wegstrecken einen Abstand von einem Block zum Ziel halten m�ssen. Es gibt einen zentralen 9x9-Block der in der Mitte das Ziel und nur an einer anderen Koordinate keine Wand enth�lt, um diese Eigenschaft zu sichern.
\\
Der Labyrinthgenerator erzeugt nach den Regeln f�r diese Labyrinthe eine Anzahl von Textdateien, die kodiert das Labyrinth beschreiben. Je eine dieser Textdateien wird bei der Generierung der Minecraft-Welt eingelesen und es wird in Minecraft das entsprechende Labyrinth f�r den Bot aufgebaut. Die Abarbeitung mehrerer Labyrinthe soll dabei z.B. durch ein Script geschehen, das nacheinander alle erzeugten Labyrinthe in Minecraft vom gegebenen Bot abarbeiten l�sst. Eine parallele Abarbeitung ist denkbar, aber noch nicht konzeptioniert.


\subsection{Optionale Anforderungen}


\section{Arbeitspakete}
% Hier sollen die einzelnen Arbeitspakete, in denen das Projekt abgearbeitet werden soll beschrieben werden.
% F�r jedes Arbeitspaket muss eine Beschreibung sowie der gesch�tzte Arbeitsaufwand in Mannstunden angegeben werden.
% Wird das Projekt von mehreren Personen bearbeitet, sollen jedem Arbeitspaket ein oder mehrere Zust�ndige Bearbeiter zugeordnet werden.
% Zus�tzlich zu den Arbeitspaketen sollen Meilensteine definiert werden. 
% Dies sind Zeitpunkte zu denen ein Zwischenergebnis bei der Projektbearbeitung erreicht wurde und eine neue Phase beginnt.

\subsection{Projektplanung}

\subsection{Informationssammlung/-zusammenstellung}


\subsection{Bedienkonzept}

\subsection{Vorversuche}

\subsection{Hardware-Entwurf}

\subsection{Software-Entwurf}

\subsection{Hardware-Aufbau}

\subsection{Software-Implementierung}

\subsection{Test des Ger�tes}

\subsection{Fehlerbehebung}

\subsection{Dokumentation}

\section{Meilensteine}
% Hier sollen Meilensteine beschrieben werden.
% Dies sind Zeitpunkte zu denen ein Zwischenergebnis bei der Projektbearbeitung erreicht wurde.

\section{Projektplan}
% Im Projektplan werden die Arbeitspakete in der realen Zeit angeordnet.
% Direkte Abh�ngigkeiten zwischen ihnen werden sichtbar gemacht und 
% 


\subsection{Arbeitsbelastung}

\end{document}


